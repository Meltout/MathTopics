\documentclass[a4paper,12pt,fleqn]{article}
\usepackage[utf8]{inputenc}
\usepackage[T1,T2A]{fontenc}
\usepackage[english,bulgarian]{babel}
\usepackage{amsthm,amsfonts,amsmath,amssymb,bbm}
\usepackage{fullpage}
\usepackage{adjustbox, graphicx}
\graphicspath{{images/AffineVarieties/}}
\usepackage{enumitem}
\setlist[enumerate]{label=\alph*.,itemsep=1pt, topsep=0pt}

\newcounter{problem}
\newcommand\problem{%
  \stepcounter{problem}%
  \textbf{Задача \theproblem.}~%
}
\newcommand\solution{%
  \textbf{Решение:}~%
}
\parindent 0in
\parskip 1em
\title{Афинни разновидности}
\author{Антъни Господинов}
\date{}

\begin{document}
    \maketitle
    \textbf{Дефиниция.~} Нека \( k \) е поле и \( f_{1},\dots,f_{s} \) са полиноми от \( k \left[ x_{1},\dots,x_{n} \right] \). Дефинираме множеството: 
    \begin{equation*}
        \mathbf{V}\left( f_{1},\dots,f_{s} \right)=\left\{ \left( a_{1},\dots,a_{n} \right) \in k^{n} \mid f_{i}(a_{1},\dots,a_{n})=0 \textit{ за всички } 1 \leq i \leq s \right\}
    \end{equation*}
    Ще наричаме \( \textbf{V}\left( f_{1},\dots,f_{s} \right) \) \textbf{афинната разновидност} определена от \( f_{1},\dots,f_{s} \). Казано по друг начин, \( \textbf{V}\left( f_{1},\dots,f_{s} \right) \) е множеството от решенията на системата уравнения: 
    \begin{equation*}
        f_{1}(x_{1},\dots,x_{n}) = \dots = f_{s}(x_{1},\dots,x_{n}) = 0
    \end{equation*} 

    \problem{Да се скицират графиките на следните афинни разновидности в \( \mathbb{R}^{2} \):
    \begin{enumerate}
        \item \( \mathbf{V}(x^{2}+4y^{2}+2x-16y+1) \).
        \item \( \mathbf{V}(x^{2}-y^{2}) \).
    \end{enumerate}
    }

    \solution{
        \begin{enumerate}
            \item 
                \begin{minipage}[t]{\linewidth}
                    \raggedright
                    \adjustbox{valign=t}{%
                    \includegraphics[scale=0.4]{variety2.1a.pdf}%
                    }
                \end{minipage}
                \item 
                \begin{minipage}[t]{\linewidth}
                    \raggedright
                    \adjustbox{valign=t}{%
                    \includegraphics[scale=0.4]{variety2.1b.pdf}%
                    }
                \end{minipage}
        \end{enumerate} 
    }

    \problem{Да се скицират графиките на следните афинни разновидности в \( \mathbb{R}^{3} \)
    \begin{enumerate}
        \item \( \mathbf{V}(x^{2}+y^{2}+z^{2}-1) \).
        \item \( \mathbf{V}(x^{2}+y^{2}-1) \).
        \item \( \mathbf{V}(x+2,y-1.5,z) \).
        \item \( \mathbf{V}(xz^{2}-xy) \).
        \item \( \mathbf{V}(x^{4}-zx,x^{3}-yx) \).
    \end{enumerate}}

    \solution{
        \begin{enumerate}
                \item 
                \begin{minipage}[t]{\linewidth}
                    \raggedright
                    \adjustbox{valign=t}{%
                    \includegraphics[scale=0.4]{variety2.4a.pdf}%
                    }
                \end{minipage}
                \item 
                \begin{minipage}[t]{\linewidth}
                    \raggedright
                    \adjustbox{valign=t}{%
                    \includegraphics[scale=0.4]{variety2.4b.pdf}%
                    }
                \end{minipage}
                \item 
                \begin{minipage}[t]{\linewidth}
                    \raggedright
                    \adjustbox{valign=t}{%
                    \includegraphics[scale=0.4]{variety2.4c.pdf}%
                    }
                \end{minipage}
                \item 
                \begin{minipage}[t]{\linewidth}
                    \raggedright
                    \adjustbox{valign=t}{%
                    \includegraphics[scale=0.4]{variety2.4d.pdf}%
                    }
                \end{minipage}
                \item 
                \begin{minipage}[t]{\linewidth}
                    \raggedright
                    \adjustbox{valign=t}{%
                    \includegraphics[scale=0.7]{variety2.4e.pdf}%
                    }
                \end{minipage}
        \end{enumerate} 
    }

    \problem{
        Да се докаже, че ако \( f \in \mathbb{C}\left[ x_{1},\dots,x_{n} \right] \) изчезва във всяка точка от \( \mathbb{Z}^{n} \) то \( f \) е тъждествено нулевия полином.
    }

    \solution{Ще докажем твърдението с индукция по броя на променливите на \( f \). В случая \( n = 1 \) от основната теорема на алгебрата знаем че ако \( f \) не е тъждествено нулевия полином, то \( f \) има най-много \( \deg(f) \in \mathbb{N} \) различни корена, от където следва че \( f \) е нулевия полином. Нека предположим, че твърдението е вярно за полиномите на най-много \( n \) променливи и нека \( f \) е полином на \( n + 1 \) променливи, \( f \in \mathbb{C}\left[ x_{1},\dots,x_{n+1} \right] \) които изчезва във всяка точка от \( \mathbb{Z}^{n+1} \). Можем да запишем \( f \) като полином на променливата \( x_{n+1} \) с коефициенти от \( \mathbb{C}\left[ x_{1},\dots,x_{n} \right] \) 
    \begin{equation}\label{eq:fwithgcoef}
        f\left( x_{1},\dots,x_{n},x_{n+1} \right) = \sum_{k=0}^{m}g_{k}\left( x_{1},\dots,x_{n} \right)x_{n+1}^{k}
    \end{equation}
    За произволни \( a_{1},\dots,a_{n} \in \mathbb{Z} \) полиномът
    \begin{equation*}
        f_{a_{1},\dots,a_{n}}\left( x_{n+1} \right) = f\left( a_{1},\dots,a_{n},x_{n+1} \right) = \sum_{k=0}^{m}g_{k}\left( a_{1},\dots,a_{n} \right)x_{n+1}^{k}
    \end{equation*}
    е полином на една променлива \( x_{n+1} \), който изчезва във всяко цяло число, тъй като \( f\left( x_{1},\dots,x_{n},x_{n+1} \right) \) изчезва във всяка точка от \( \mathbb{Z}^{n+1} \). По индукционно предположение това значи, че \( f_{a_{1},\dots,a_{n}} \) е тъждествено нулевия полином, откъдето следва че коефициентите му \(g_{0}\left( a_{1},\dots,a_{n} \right),\dots, g_{m}\left( a_{1},\dots,a_{n} \right)\) се анулират. Тъй като \( a_{1},\dots,a_{n} \in \mathbb{Z}\) бяха произволни, то значи \( g_{0},\dots,g_{m} \) изчезват във всяка точка от \( \mathbb{Z}^{n} \) и от индукционното ни предположение следва че \( g_{0},\dots,g_{m} \) са тъждествено нулевия полином. От представянето \eqref{eq:fwithgcoef} на \( f \) като полином на \( x_{n+1} \) с коефициенти \( g_{0},\dots,g_{m} \) имаме: 
    \begin{equation*}
        f\left( x_{1},\dots,x_{n},x_{n+1} \right) = \sum_{k=0}^{m}g_{k}\left( x_{1},\dots,x_{n} \right)x_{n+1}^{k} = \sum_{k=0}^{m}0\,x_{n+1}^{k}=0
    \end{equation*}
    което доказва, че \( f \) е тъждествено нулевия полином. \hfill \( \qed\)
    }
\end{document}