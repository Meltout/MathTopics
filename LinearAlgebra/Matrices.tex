\documentclass[a4paper,12pt,fleqn]{article}
\usepackage[utf8]{inputenc}
\usepackage[T1,T2A]{fontenc}
\usepackage[english,bulgarian]{babel}
\usepackage{amsthm,amsfonts,amsmath,amssymb,bbm}
\usepackage{fullpage}
\usepackage{enumitem}
\usepackage{url}

\setlist[enumerate]{label=\alph*.,itemsep=1pt, topsep=0pt}

\newcommand{\rvline}{\hspace*{-\arraycolsep}\vline\hspace*{-\arraycolsep}}

\newcounter{problem}
\newcommand\problem{%
  \stepcounter{problem}%
  \textbf{Задача \theproblem.}~%
}
\newcommand\solution{%
  \textbf{Решение:}~%
}
\parindent 0in
\parskip 1em
\title{Действия с матрици}
\author{Антъни Господинов}
\date{}

\begin{document}
    \maketitle{}

    \problem{Да се пресметне произведението \( AB \) за: 
      \begin{enumerate}
        \item 
        \( A = 
          \begin{bmatrix}
            1 & 2 & 3 \\
            3 & 3 & 1
          \end{bmatrix}
        \),
        \( B = 
          \begin{bmatrix}
            -8 & -4 \\
             \phantom{-}9 & \phantom{-}5  \\
            -3 & -2
          \end{bmatrix}
        \)
        \item 
        \( A = 
           \begin{bmatrix}
              1 & 4 \\
              1 & 2
           \end{bmatrix}
        \),
        \( B = 
           \begin{bmatrix}
              6 & -4 \\
              3 & \phantom{-} 2
           \end{bmatrix}
        \)
      \end{enumerate}
    }

    \problem{
      Нека 
      \( A = \left[ a_{1},\dots,a_{n} \right] \) е вектор ред, а 
      \( B = \begin{bmatrix}
        b_{1} \\
        \vdots \\
        b_{n}
      \end{bmatrix} \) е вектор стълб. Да се пресметнат произведенията \( AB \) и \( BA \)
    }

    \problem{
      Нека \( A, B \) и \( C \) са матрици с размери \( l \times m, m \times n \) и \( n \times p \). Колко умножения са нужни за да пресметнем произведението \( AB \)? В какъв ред трябва да пресметнем тройното произведение \( ABC \), ако искаме да минимизираме броя на умноженията?
    }
    
    \problem{
      Пресметнете \( 
        \begin{bmatrix}
          1 & a \\
          0 & 1
        \end{bmatrix}
        \begin{bmatrix}
          1 & b \\
          0 & 1
        \end{bmatrix}  
      \) и \(
        \begin{bmatrix}
          1 & a \\
          0 & 1
        \end{bmatrix}^{n} 
      \)
    }

    \problem{
      Намерете формула за \(
        \begin{bmatrix}
          1 & 1 & 1 \\
          0 & 1 & 1 \\
          0 & 0 & 1
        \end{bmatrix}^{n} \) и я докажете с индукция.
    }

    \problem{Пресметнете следните произведения чрез блочно умножение:\\
    (\url{https://en.wikipedia.org/wiki/Block_matrix#Block_matrix_multiplication})
        \begin{center}
        \(
          \begin{bmatrix}
            \begin{matrix}
              1 & 1 \\
              0 & 1
            \end{matrix}
            & \rvline &
            \begin{matrix}
              1 & 5 \\
              0 & 1
            \end{matrix} \\
            \hline
            \begin{matrix}
              1 & 0 \\
              0 & 1
            \end{matrix}
            & \rvline &
            \begin{matrix}
              0 & 1 \\
              1 & 0
            \end{matrix}
          \end{bmatrix} \)
          \(\begin{bmatrix}
            \begin{matrix}
              1 & 2 \\
              0 & 1
            \end{matrix}
            & \rvline &
            \begin{matrix}
              1 & 0 \\
              0 & 1
            \end{matrix} \\
            \hline
            \begin{matrix}
              1 & 0 \\
              0 & 1
            \end{matrix}
            & \rvline &
            \begin{matrix}
              0 & 1 \\
              1 & 3
            \end{matrix}
          \end{bmatrix}
        \),
        \(
          \begin{bmatrix}
            0
            & \rvline &
            \begin{matrix}
              1 & 2
            \end{matrix} \\
            \hline 
            \begin{matrix}
              0 \\
              3
            \end{matrix}
            & \rvline & 
            \begin{matrix}
              1 & 0 \\
              0 & 1
            \end{matrix}
          \end{bmatrix}
        \)
        \( 
          \begin{bmatrix}
            1
            & \rvline &
            \begin{matrix}
              2 & 3
            \end{matrix} \\
            \hline 
            \begin{matrix}
              4 \\
              5
            \end{matrix}
            & \rvline & 
            \begin{matrix}
              2 & 3 \\
              0 & 4
            \end{matrix}
          \end{bmatrix}
         \)
        \end{center}
    }

    \problem{
      Нека \( D \) e диагонална матрица с елементи \( d_{1},\dots,d_{n} \) и нека \( A = \left( a_{ij} \right) \) е произволна \( n \times n \) матрица. Пресметнете произведенията \( DA \) и \( AD \). 
    }

    \problem{
      Докажете, че произведението на две горнотриъгълни матрици също е горнотриъгълна матрица.
    }

    \problem{
      Казваме, че матриците \( A \) и \( B \) комутират, ако \( AB = BA \). Да се намерят всички матрици, които комутират с:
        \begin{enumerate}
          \item \( 
            \begin{bmatrix}
              1 & 0 \\
              0 & 0
            \end{bmatrix} \)
          \item \( 
            \begin{bmatrix}
              0 & 1 \\
              0 & 0
            \end{bmatrix} \)
          \item \( 
            \begin{bmatrix}
              1 & 3 \\
              0 & 1
            \end{bmatrix} \)
        \end{enumerate}
    }
    \problem{
      \textit{Следата} на квадратна матрица е сумата на елементите по диагонала: 
      \begin{center}
        \( tr(A) = a_{11} + a_{22} + \dots + a_{nn} \)
      \end{center}
      Докажете, че \( tr(A+B) = tr(A) + tr(B) \) и също
       \( tr(AB) = tr(BA) \)
    }

    \problem{
      Докажете, че уравнението \( AB - BA = E \) няма решение в пространството на квадратните \( n \times n \) матрици с елементи от полето на реалните числа(\textit{понякога това пространство се означава като: \( M_{n,n}\left( \mathbb{R} \right) \)})
    }

\end{document}