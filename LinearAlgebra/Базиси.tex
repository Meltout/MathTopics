\documentclass[a4paper,12pt,fleqn]{article}
\usepackage[utf8]{inputenc}
\usepackage[T1,T2A]{fontenc}
\usepackage[english,bulgarian]{babel}
\usepackage{amsthm,amsfonts,amsmath,amssymb,bbm}
\usepackage{fullpage}
\usepackage{esvect}

\newcounter{problem}
\newcommand\problem{%
  \setcounter{equation}{0}
  \stepcounter{problem}%
  \textbf{Задача \theproblem.}~%
}
\newcommand\solution{%
  \textbf{Решение:}~%
}
\parindent 0in
\parskip 1em
\title{Базиси, координати и размерност в крайномерни линейни пространства}
\author{Антъни Господинов}
\date{}

\begin{document}
\maketitle

    \textbf{Фундаментален факт:}~%
      Да се докаже, че ако \( \textbf{S}=\left( s_{1},\dots,s_{n} \right) \) е система от линейно независими вектори в линейно пространство, а \( \textbf{T}=\left( t_{1},\dots,t_{m} \right) \) е линейно независима система вектори в същото линейно пространство, които се съдържат в линейната обвивка на \textbf{S}, то \( n \geq m \)
    

    \textbf{Доказателство:}~%
      (Матрична перспеткива)
      Всеки вектор от \textbf{T} може да се изрази като линейна комбинация на вектори от \textbf{S} по следния начин:
      \begin{equation} \label{eq:1}
        t_{i}=\lambda_{1,i} s_{i} + \dots + \lambda_{n,i} s_{n}
      \end{equation}
      Тривиално е да се провери че условието е вярно за \( n=1 \).
      Ще разгледаме случая, в който \( n > 1 \). Да допуснем, че \( n<m \). Векторите \( (t_{1},\dots,t_{m}) \) са линейно зависими тогава и само тогава когато съществува тяхна нетъждествено нулева линейна комбинация равна на нулевия вектор на пространството:
      \begin{align} \label{eq:2}
        \alpha_{1}t_{1}+\dots+\alpha_{m}t_{m}=\vv{0} && \left( \alpha_{1},\dots,\alpha_{m} \right) \neq \left( 0,\dots,0 \right)
      \end{align} 
      Ползвайки \eqref{eq:1} и \eqref{eq:2} може да изразим условието за линейна зависимост на \( t_{1},\dots,t_{m} \) като това следната хомогенна система от линейни уравнения да има нетривиално решение:
      \begin{gather}\label{eq:3}
        \begin{pmatrix}
          \lambda_{1, 1} & \lambda_{1, 2} & \lambda_{1, 3} & \dots & \lambda_{1, m} \\
          \lambda_{2, 1} & \lambda_{2, 2} & \lambda_{2, 3} & \dots & \lambda_{2, m} \\
          \vdots & \vdots & \vdots & \ddots & \vdots \\
          \lambda_{n-1, 1} & \lambda_{n-1, 2} & \lambda_{n-1, 3} & \dots & \lambda_{n-1, m}  \\
          \lambda_{n, 1} & \lambda_{n, 2} & \lambda_{n, 3} & \dots & \lambda_{n, m} 
        \end{pmatrix}
        \begin{pmatrix}
          \alpha_{1} \\
          \alpha_{2} \\
          \vdots \\
          \alpha_{m}
        \end{pmatrix}
        =
        \begin{pmatrix}
          0 \\
          0 \\
          \vdots \\
          0
        \end{pmatrix}
      \end{gather}
      Ще докажем съществуването на това нетривиално решение чрез индукция. Тъй като \( \textbf{T} \) е линейно независима система вектори, то \( t_{m} \neq \vv{0} \) и съществува индекс \( j \) с \( \lambda_{j,m} \neq 0 \). Без ограничение на общността може да приемем, че \( j = n \) и съответно \( \lambda_{n, m} \neq 0 \).
      След добавяне на подходящи кратни на ред \( n \) към останалите редове на матрицата от \eqref{eq:3} с цел елиминация на елементите в колона \( m \) получаваме следното еквивалентно матрично уравнение:
      \begin{gather}\label{eq:4}
        \begin{pmatrix}
          \mu_{1, 1} & \mu_{1, 2} & \mu_{1, 3} & \dots & 0 \\
          \mu_{2, 1} & \mu_{2, 2} & \mu_{2, 3} & \dots & 0 \\
          \vdots & \vdots & \vdots & \ddots & \vdots \\
          \mu_{n-1, 1} & \mu_{n-1, 2} & \mu_{n-1, 3} & \dots & 0 \\
          \lambda_{n, 1} & \lambda_{n, 2} & \lambda_{n, 3} & \dots & \lambda_{n, m} 
        \end{pmatrix}
        \begin{pmatrix}
          \alpha_{1} \\
          \alpha_{2} \\
          \vdots \\
          \alpha_{m}
        \end{pmatrix}
        =
        \begin{pmatrix}
          0 \\
          0 \\
          \vdots \\
          0
        \end{pmatrix}
      \end{gather}
      Матрицата вече има вид, които ни позволява да приложим индукция. От ред \( n \) може да изразим \( \alpha_{m} \):
      \begin{equation*}
        \alpha_{m}=\frac{\lambda_{n,1}\alpha_{1}+\dots+\lambda_{n,m-1}\alpha_{m-1}}{-\lambda_{n,m}}
      \end{equation*}
      от където следва че решенията на \eqref{eq:4} са в биективно съответсвие с решенията на системата:
      \begin{gather}\label{eq:5}
        \begin{pmatrix}
          \mu_{1, 1} & \mu_{1, 2} & \mu_{1, 3} & \dots & \mu_{1,m-1} \\
          \mu_{2, 1} & \mu_{2, 2} & \mu_{2, 3} & \dots & \mu_{2,m-1} \\
          \vdots & \vdots & \vdots & \ddots & \vdots \\
          \mu_{n-1, 1} & \mu_{n-1, 2} & \mu_{n-1, 3} & \dots & \mu_{n-1,m-1}
        \end{pmatrix}
        \begin{pmatrix}
          \alpha_{1} \\
          \alpha_{2} \\
          \vdots \\
          \alpha_{m-1}
        \end{pmatrix}
        =
        \begin{pmatrix}
          0 \\
          0 \\
          \vdots \\
          0
        \end{pmatrix}
      \end{gather} 

      Важно е да отбележим, че колоните на матрицата \eqref{eq:5} са линейно независими, което е ключовият факт, който ни позволява да приложим така нареченият похдод \textit{структурна индукция}. Интуитивно, по-малката матрица притежава структурните свойства на голямата, от които сме се възползвали. В нашия случай единственото свойство, от което се възползвахме е това, че колоните на голямата матрица са линейно независими. \textit{Доказателството на този факт оставям като задача за любознателния читател.}

      Явният вид на съответсвието е:
      \begin{equation*}
        \left( \alpha_{1},\dots,\alpha_{m-1} \right) \longmapsto \left( \alpha_{1},\dots,\alpha_{m-1}, \frac{\lambda_{n,1}\alpha_{1}+\dots+\lambda_{n,m-1}\alpha_{m-1}}{-\lambda_{n,m}}\right)
      \end{equation*}

      От където веднага може да съобразим че \eqref{eq:4} има нетривиално решение тогава и само тогава когато \eqref{eq:5} има нетривиално решение. \\
      Страхотно за нас, тъй като \eqref{eq:5} има нетривиално решение по индукционно предположение.
      \qed

    \textbf{Следствие:~} Всеки два базиса на крайномерно линейно пространство имат равен брой елементи.

    \newpage

    \problem{
      Докажете, че множеството \( \textbf{B}=\{(1, 2, 0)^{t}, (2, 1, 2)^{t}, (3, 1, 1)^{t}\} \) е базис на \( \mathbb{R}^{3} \) и намерете координатите на вектора \( \vv{v}=(1, 2, 3)^{t} \) спрямо този базис.
    }

    \problem{
      Намерете базиси за подпространствата \( \textit{S}=\{\, (a, b, c, d) \mid a + c + d = 0 \,\} \) и \( {\textit{T}}=\{\, (a, b, c, d) \mid c = 2d \,\} \) на \( \mathbb{Q}^{4} \). Каква е размерността на сечението \( \textit{S} \cap \textit{T} \)?
    }

    \problem{
      Намерете базис на подпространството \( \textit{U}=\left\{\, p(x) \mid p(1)=0 \, \right\} \) на пространството \( \mathbb{Q}^{3}[x] \) на полиномите с рационални коефициенти от степен по-малка или равна на 3.
    }

    \problem{
      Нека \( \textit{W} \subset \mathbb{R}^{4} \) е пространството от решенията на системата линейни уравнения \( AX=0 \), където
      \( A=\begin{bmatrix}
        2 & 1 & 2 & 3 \\
        1 & 1 & 3 & 0
      \end{bmatrix} \). Намерете базис на \textit{W}.
    }

    \problem{
      Нека \( \vv{v}=(v_{1}, v_{2}, v_{3}, v_{4}) \) е вектор от \( \mathbb{R}^{4} \). Координатите на \( \vv{v} \) могат да се разбъркат по 24 различни начина, например като \( (v_{2}, v_{1}, v_{3}, v_{4}) \) или \( (v_{3}, v_{1}, v_{4}, v_{2}) \) и по този начин да получим други вектори от \( \mathbb{R}^{4} \). Линейната обвивка на тези 24 вектора е подпространство \textit{S} на \( \mathbb{R}^{4} \). Да се намери конкретен вектор \( \vv{v} \), за който размерността на \textit{S} е: \\
      a) едно \\
      б) две \\
      в) три \\
      г) четири 
    }

    \problem{
      Нека \textit{V} и \textit{W} са подпространства на \( \mathbb{R}^{n} \), изпълняващи неравенството:
      \begin{equation*}
         dim(\textit{V}) + dim(\textit{W}) > n
      \end{equation*}
      Да се докаже, че съществува ненулев вектор в сечението на \textit{V} и \textit{W}.
    }

    \problem{
      Нека \textit{A} е \( \textit{n} \times \textit{m} \) матрица и нека \( \textit{A}^{\prime} \) е получена чрез прилагане на поредица от елементарни преобразования по редове(\textit{това са преобразованията, които правим в метода на Gauss-Jordan}) върху \textit{A}. Да се докаже че линейната обвивка на вектор-редовете на \textit{A} съвпада с линейната обвивка на редовете на \( \textit{A}^{\prime}\).
    }

\end{document}